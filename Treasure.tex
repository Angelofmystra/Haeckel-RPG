
\section{Random Treasure Generation}
\begin{multicols}{2}



Treasure will \textbf{usually be found in the lairs of monsters}. Each monster entry in Chapter 8, Monsters, designates whether or not that type of monster has the possibility of treasure. Those that do are given a Treasure Type (TT), which indicates what kinds and quantities of treasure a monster might have.

The Treasure Types are lettered from A to R, with TT A yielding the smallest hoards and TT R the largest. To randomly generate a monster’s treasure, the Judge finds the row on the Treasure Type table, below, that corresponds to the monster’s TT. For each column on the row, the Judge rolls the appropriate dice to determine whether the specified treasure is present, and if so, in what quantity. When the dice indicate that gems, jewelry, or magic items are present, the specific treasures found in these categories are then determined using sub-tables provided after the Treasure Type table. If the presence of magic items is indicated, but no specific type is indicated, the type is determined by rolling on the Random Magic Type table. Judges wishing to create more diverse and interesting treasures may then combine the treasure generated into lots and use the Special Treasures table, as described below.

\paragraph{Dungeon Treasure Distribution} The Treasure Type table is designed to place a total amount of treasure in any given lair or dungeon equal to four times the XP value of the monsters in the area. Thus, on average, the amount of treasure assigned to a monster by its Treasure Type equals four times its XP. However, there is a correction factor that compensates for encounters with monsters that have no treasure by assigning other types of monsters more treasure. For example, in a dungeon level 2 stocked with the default random monster tables, over half the encounters will be with monsters like zombies and giant bats that do not carry treasure. The remaining dungeon level 2 monsters that do carry treasure therefore have been assigned treasure worth eight times their XP value. In this way, the overall gp to XP ratio within that set of monsters remains approximately 4:1.

\paragraph{Monster Types} In addition to their quantitative rank, the Treasure Types are further sub-divided into three categories, representing the method by which the monsters have accumulated the treasure: by hoarding, by raiding, or incidentally. 

\paragraph{Hoarding monsters} are typically intelligent enough to appreciate the value of wealth, and powerful enough to gain and defend substantial amounts of it. Hoarder Treasure Types (B, D, H, N, Q, R) feature a roughly equal mix of coins and other valuables, and typically yield values far in excess of their weight (usually 10 stone or less).

\paragraph{Raiding monsters} are intelligent creatures who gather treasure by stealing it from those weaker than themselves. Raiders are often the minions of more powerful monster who claim the most precious items for themselves, so Raider Treasure Types (E, G, J, L, O) are mostly bulky hoards (typically 20 stone) of low value coin and goods.

\paragraph{Incidental Hoaders} Monsters with incidental treasure accumulate wealth accidentally, e.g. by bringing dead adventurers back to the lair to feed their young. Though they may lack the intelligence to recognize things of value, they may be attracted to shiny objects which they gather as best they can with their claws or jaws. The Incidental Treasure Types (A, C, F, I, K, M, P) are usually only a few stone in weight and are the most variable type of treasure, with some rolls on these tables yielding fantastic bounty while as many others produce nothing at all.

A dungeon with a variety of monsters will thus tend to have a wide assortment of interesting treasures, ranging from precious regalia to bulky low value coin.

Unless otherwise noted, randomly generated treasures are found only in the monster’s lair. If the monster entry specifies that a creature carries individual treasure either instead of or in addition to its lair treasure, these individual treasures may be found even if the monster is encountered when wandering outside its lair.
\end{multicols}
\paragraph{Gem Values}

\paragraph{Gem Quality}
\begin{tabular}{l | l l l l}
    & Raw & Flawed & Cut & Pattern \\
    \hline
    Agate & & & & \\
    Alexandrite & & & & \\
    Amber & & & & \\
    Amethyst & & & & \\
    Aquamarine & & & & \\
    Adventurine & & & & \\
    Bloodstone & & & & \\
    Carnelian & & & & \\
    Citrine & & & & \\
    Coral & & & & \\
    Diamond, blue & & & & \\
    Diamond, yellow & & & & \\
    Diamond, crimson & & & & \\
    Emerald & & & & \\
    Garnet & & & & \\
    Jade, blue & & & & \\
    Jade, crimson & & & & \\
    Jade, green & & & & \\
    Jasper & & & & \\
    Kunzite & & & & \\
    Lapis Lazuli & & & & \\
    Malachite & & & & \\
    Moissanite & & & & \\
    Moonstone & & & & \\
    Obsidian & & & & \\
    Onyx & & & & \\
    Opal & & & & \\
    Pearl, black & & & & \\
    Pearl, white & & & & \\
    Peridot & & & & \\
    Quartz & & & & \\
\end{tabular}


\paragraph{Jewelry Values}
