\chapter{Cities Overview}
Abyssimiar has a total of 11 cities. Those named Magna are large. Those named Minor are small. Each city is located either on the flood plains through the core of Abyssimiar or receives water through artificially crafted canals that are magnificent feats of engineering.

Puzzilingly, Lower Abyssimiar is the Northern most region on Haeckelian maps. And, conversely, Upper Abyssimiar is the Southern most region on Haeckelian maps. 

\section{Magna Cities}

\subsection{Herakleopolis Magna}

\paragraph{Architecture} Unplanned. Mix of Egyptian and Greek architecture.

\paragraph{Overview} City dedicated to the Ram God of the Lake. He who controls this city is the ruler of the riverbanks. A large part of this city is a Necropolis of Sedmet el-Gebel. It connects Abyssimiar to Ubur Furis. Puzzilingly there are influences of greek architecture - but there is no historical justification for it - the notion of an absurd history is not a perplexity for the natives.

\paragraph{Theme} Fear of the Rising Dead and Curses of wronged women.

\paragraph{Region} Lower Abyssimiar (North)

\paragraph{Regional Animal} Lions.

\subsection{Taposiris Magna}

\subsection{Memphis Magna}

\subsection{Medina Magna}

\subsection{Abydos Magna}