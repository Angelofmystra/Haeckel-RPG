\chapter{The Campaign}\label{campaign}
\newpage

\begin{multicols}{2}



The purpose of this chapter is to introduce the players to some of the core concepts to be used in the Haeckel campaign. 

One of the design mantras of Haeckel is the theme of exploration. Our house rules aim to provide this. They are to promote the idea that players will explore in game and experiment with things to see its effects. For example, finding certain herbs and brewing them to see what happens.
\begin{framed}\centering
The rules covered here are only the basic principles that will introduce the players to the style of play. The actual meat and bones for the content that the players will have to explore will be covered in a GM's Guide.
\end{framed}
\subsection{Experience}

The primary method by which experience is gained is through looting treasure from the wilderness or dungeons and bringing it back to civilisation. The full gold value of the haul determines how much experience is gained via this method. There is a 1 to 1 ratio between gp looted and xp given. 

Compared to other RPG systems, Haeckel gives a reduced amount of experience for killing foes. You will find that slaying composes of roughly one quarter of the total XP gained during an adventure however. 

If a magical item is found and is used during the adventure no XP is gained from bringing it back - this would count as doubling the benefit.

In general any armour found during an adventure - either acquired from combat or found in some fashion - will be damaged and be in need of repair. It will be valued at half price.

\subsection{Power Curve} How powerful do you start? How powerful are the monsters relative to you? How powerful are humans relative to you? How much power do you gain over time? How powerful are you in the end game? What about the monsters? 

In essence, the two things that matter is absolute strength and relative strength. Are there any powerspikes?
 
\subsection{Player Agency} 

\subsection{House Rules: Alchemy}
Alchemy is used for two purposes. Making potions and making alcohol. This system is a "grab and mix" system where the player explores to find ingredients, and then either finds recipes or experiments himself to see what he can make.

Tier 1 potions require Aqua Vitae and one other ingredient. Tier 2 potions require Aqua Vitae, another Liquid, and one ingredient. And so on. 

% Aqua Fortis
% Aqua Regia


% Heal Potion
% Spell Restore Potion
% Light Potion - High utility
% Pungent Potion - High utility
% Slime (Acids) Potion
% % Black    Eats metal
% % Green    Eats flesh
% % Brown    Eats anything but stone
% % Yellow   Eats wood 
% % Blue     Paralysis
% % White    Eats glass
% % Red      
% % ...      Depth Charge
% % ...      Sticky
% Invisibility Potion
% 

    \subsubsection{Core Ingredients}
        \paragraph{Aqua Vitae} The base liquid used in all potions. 
\subsection{House Rules: Herbalism}
\subsection{House Rules: Languages}

\subsection{Henchmen}
Henchmen are non-player characters that get a full share of the experience and treasure that the players acquire in their adventure. They provide great support to the party for many reasons. In this subsection I will cover some example Henchmen that the players will know about.

\subsubsection{The Lumberjack} The Lumberjack is truly and deeply in love, in the purest storybook sense. His love is not necessarily requited, but acts as a source of strength and purpose, for he would cross oceans and mountains to protect his beloved. She suffers from an ailment that has nearly struck her down cold but is now in a coma. He is trapped in a state of turmoil and will grasp at straws to save his love. 
%He aims to build up enough wealth so that he may afford a proven spiritual healer to restore her vitality.
\subsubsection{The Albino Terror} Hailing from the harshest lands in the howling abyss, this Albino barbarian stands at a mighty 6ft 10" tall and dual wields Abloni gladius' that he obviously looted from their cold dead hands. His monstrously ugly face and straw-like face makes him intimidating to all but the most indomitable souls. Despite his fearsome appearance, you can trust that he has a deep and strong sense of honour, a code of conduct that makes him very loyal to his allies. His hatred for Mages is widely known.
\subsubsection{The Feline Grace} A softly spoken man that is light of foot and almost impossible to hear while walking. 

\section{Thieves Guilds of Haeckel}

The interaction of the Thief and the Thieves Guild can be an important aspect of the character. The following is a standardised set of advantages that a Thief gets from joining the Thieves' Guild that operates in the territory.

\begin{itemize}
    \item Fences who will buy stolen goods as well as regular goods. They often carry larger sums of money - effectively making the market size bigger. 
    \item Awareness of Chest and Lockmaking secrets that allows the crafting of more advanced chests. This would be made very relevant to the rules by GMs who make thieves a constant threat to the players in a campaign.
    \item Are able to avoid criminals whom are loyal to the Guild - thus making travel abit safer. 
    \item Access to restricted item options, such as lockpicks, as the authorities try to limit who has access to these kinds of equipment. 
    \item At a sufficient level of City Corruption, the guards who confront you will allow you to keep Stolen Goods.
    \item Provide a means by which you can escape jail in such a way as to avoid fines or gaining a bounty on your head.
    \item Master Thieves whom are willing to take talented apprentices and teach them the secrets of the trade.
\end{itemize}

\section{Campaign Play: Crime Rules}

Once the players acquire a home-base, or maybe even a reputation for having wealth, thieves will regularly attempt to separate them from it. The frequency is dependent on their reputation - and the chance of its success depends on the player’s security measures. I would probably have some random roll determining the type of thief - such as one who burgles, or one who actively haunts the players waiting for an opportunity to strike. 

In this way, developing security measures is something players will need to make eventually. 

\section{Restricted Goods} 
\begin{itemize}
    \item Lockpicks
    \item Magical Potions
    \item Alchemical Goods
    \item Advanced Locks / Chests that even the authorities would have trouble breaking into
    \item Corpses for the purposes of necromancy or medical experimentation
    \item Drugs
    \item Poisons
    \item Smuggling equipment
    \item Provide you the means to avoid taxed travel routes.
    \item Trap mechanisms.
    \item Books restricted by oppressive authorities.
    \item Secret maps.
\end{itemize}


\end{multicols}
% Bestiary
% % Giant Eye Ball - The Gatekeeper
% % Giant Skull - Chases you while spawning enemies that try to cut off your retreat
% % Giant Fireball - Chases you, occassionally charges, 

% Anger is more useful than despair.
% Villainy wear many masks, none of which is more dangerous than virtue.
% You can always judge a man by the quality of his enemies.
% Guilt is petty bourgeois crap. An artist creates his own moral universe.
