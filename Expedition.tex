
\begin{multicols}{2}



\paragraph{Basic questions}

\begin{itemize}
    \item How far is the target location?
    \item Do you know any local guides? Is there a key benefit to asking a local guide?
    \item Do we know of any threats enroute?
    \item Do we know of any threats at the location?
    \item Are there any weather considerations?
    \item How could we carry the loot back?
    \item How much supplies do we have and need?
    \item Are there any local sources of food and water en route or at the location? Any known poisonous plants or tainted water?
\end{itemize}

\paragraph{Basic Travel Logistics} I will assume at least one character is in armour. Thus 16 miles per day. By Horse, that is roughly 32 miles per day. I suggest placing an adventure site 32 miles away. Thus it is 1 day by horse, 2 days by foot (in armour). If its trackless in forest, swamp, hills, then double the time. So 2 days by horse, 4 days by foot (in armour). Trackless in plains? 1 and a half days by horse. 3 days by foot.

\paragraph{Food} Each day the player must eat one trail rations. So a basic adventure would require from each adventurer between 1 and 4 rations, each way. So possibly between 2 to 8 each.

\paragraph{Water} A nice way to handle this is to say that a character needs either 1 or 2 gallons of water per day. The waterskin only has two gallons. The player needs 2 gallons if he was in a high stress situation, such as combat, thus increasing his water need. So if each player has only a single waterskin each, then there will not be enough for the return trip, or the adventure at the adventure site. Thus they need to identify local sources of good water (or carry more waterskins, which is quite heavy).

\paragraph{Additional Note} I chose to use the above system because it allows the players to make a strategic decision on whether to fight or not, based on how much water they have. My system assumes normal climate conditions, and normal clothing for the area. For deserts such as Abyssimiar, you can double the water requirement.

\paragraph{Start Travel Time and its effect on the Adventure} If the players travel by day, and its 1 day (actually 8 hours overland) by horse, then they will reach the adventure site at night time. This means they will be tired, and must rest. If they dont rest, they will suffer tired rules. If they sleep, then they will be ready to infiltrate the adventure site by the early morning hours. This means that stealth would be difficult for the players to use - and ranged combat would dominate the fighting.

If the players travel by night, and its 1 day (actually 8 hours overland) by horse, then they will reach the adventure site by day time. They will be tired at this point. It is generally unsafe to set up camp near the adventure site - if they have roaming patrols. Thus they must set up camp some distance away and sleep. This means the players will be awake and ready at night time (or late evening) and be capable of easily using the darkness possibly to their advantage.

Any cooking of food will most likely alert sentries to the players position. This could apply a morale penalty to henchmen/followers whom are unused to such excursions. I have personally seen how city dwellers react to this - it pisses people off and slightly breaks them. The promise of alcohol in the evening will negate this penalty.

\paragraph{Hard Core Mode} If you travel by road, you will be seen by sentries, watch towers, and will have to deal with the possibility bandits (whom will thus require you to burn up 2 gallons of water each due to the strain). What are the penalties of being seen? Well, no chance of being surprised, and a good chance you will be surprised.

If you travel by trails, you will be capable of avoiding the above problems - but you now travel at 3/4 speed: so it was as if you was travelling trackless in plains (see above).

Every time the players go to a water source, there is a chance of a random encounter (boars are quite deadly, and natural inhabitants of most watery places).

Roll at least once per day for a random encounter. The more time the players spend, the more it becomes difficult to accomplish the mission.

Have a group have at least one intelligent enemy whom is a leader of some sort. He will for sure use sound tactics against the players - attacking the players from the right flank to negate shields, creating ambush zones, employing defensive terrain (such as hills, pits, water) with weapons appropriately, ensures his troops are properly fitted with ranged weapons and sufficient ammo. These are some of the basic tactics employable.

\end{multicols}

\paragraph{Dungeon Play Sequence}
\begin{itemize}
    \item Turns progress in the labyrinth as characters use turns to move about, look for traps, listen for noises, or search areas.
    \item GM occassionally rolls for random encounters. If something is encountered (either planned, or random) then shift to encounter time.
    \item At this point roll 2d6x10 to determine the distance in feet separating the characters from the monster. Next the GM determines if either side or both is surprised. Then roll to determine initiative (1d6). Then a 2d6 is rolled to determine the monster's reaction.
\end{itemize}


\paragraph{Wilderness Play Sequence}
\begin{itemize}
    \item roll d\% at the start of each day to determine if the players are lost.
    \item continue as you would a Dungeon except that monsters encountered will start a distance of 4d6x10 yards away. 
\end{itemize}


