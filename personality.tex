\newpage
\chapter{Personality}\label{personality}
\pagecolor{gray}\afterpage{\nopagecolor}
\newpage
\pagecolor{gray}\afterpage{\nopagecolor}
blah blah
\newpage

\changepage{9cm}{9.4cm}{-4.7cm}{-4.7cm}{}{-4.5cm}{}{}{}
%\noindent\rule{\textwidth}{\textheight}
\includegraphics[width=\textwidth,height=\textheight]{Avalonia4}
\newpage

%restoring the standard settings
\changepage{-9cm}{-9.4cm}{4.7cm}{4.7cm}{}{4.5cm}{}{}{}

\begin{multicols}{2}

\subsection{Ethos}
    \paragraph{Good}
    \paragraph{Evil}

\subsection{Spheres}
Think of your sphere as your birth sign, with power over your deepest thoughts. The reality is that spheres are aspects made manifest by the influences of the esoteric Dread Gates. It is suggested to use Spheres as a way to theme your character and influence his narrative. 

Dont think of spheres as a way to straightjacket your character. They provide no obvious mechanical benefit. It is upto the player to decide how the sphere influences and shapes them. It is possible that multiple spheres are relevant to your character. You have a choice of one sphere and this acts as the primary sphere. The rest would be secondary. 

One must remember that the definitions of the words in the Spheres are not the modern ones that most readers are used to. We have chosen to use the anarchronistic approach to them. The spheres are - in its true form - not a moral compass of Humanity but actually supernatural or elemental concepts; though certainly it can define (or promote) the moral compass of a character. Accordingly, we have listed the definitions of words that we have used that are not obvious to the modern reader.

\begin{tabular}{l|c|r}
    Emotions & Elements & Spiritual \\
    \hline
    Love & Fire & Blasphemy\\
    Hate & Water & Apostacy\\
    Fear & Earth & Despair\\
    Pride & Air & Hatred\\
    Greed & Lightning & Acedia\\
    Lust & Wood & Envy\\
    Apathy & Metal & Sloth\\
    Revenge & Light & Wrath\\
    Honour & Darkness & Gluttony\\
    Glory & Storms & \\
\end{tabular}

\paragraph{Apathy} Refusal to help others in need.
\paragraph{Acedia} Lack of joy towards the World.
\paragraph{Sloth} As related to laziness with regards to work or spiritual laziness.
\paragraph{Vengeance} Love of justice perverted to revenge and spite.

\subsection{World View}
\subsubsection{Philosophies}
    \paragraph{Stoicism} "The greatest good was contentment and serenity."
    \paragraph{Hedonism} "Eat, drink and be merry, for tomorrow we die."
    \paragraph{State consequentialism} The moral worth of an action based on how much it contributes to the basic goods of a state.
\paragraph{Ideologies}
\subsection{Politics}
\paragraph{Imperial vs Provincial}
\paragraph{Monarchy vs Republic}
\paragraph{Secular vs Theocratic}
\paragraph{Class conflict}

\end{multicols}