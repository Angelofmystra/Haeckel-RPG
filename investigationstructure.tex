\chapter{Investigation Structure}

\section{PHASE 1 – Identifying the Threat}
\begin{itemize}
    \item Townsfolk (complication: acquiring their trust)
    \item Folklore
    \item The Gypsies (complication: locating them)
    \item Physical evidence
\end{itemize}

\section{PHASE 2 -- Researching the foe}
In essence you are determining the best means to face it. This involves:
\begin{itemize}
    \item Collecting information (complication: attention seekers, or accomplices)
    \item Whatever information about its behaviour, past and present
    \item Actual witnesses to the creature
    \item If nobody has actually seen it, any strange and unexplainable occurrences have been witnessed recently
    \item Learn what individuals have recently settled in the community, who has recently left (or vanished), and who has recently died.
    \item Hypothesis
    \item Observing the creature (complication: it observes you back)
\end{itemize}

\section{PHASE 3 -- Formulating the Plan}
\begin{itemize}
    \item Is Combat Necessary?
    \item Play the Fiend’s Advocate
    \item Draw the Creature Out
    \item Prepare for Escape
\end{itemize}




